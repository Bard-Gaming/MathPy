\documentclass[a4paper, 12pt]{article}
\usepackage{amsfonts}
\usepackage{amssymb}

\title{\vspace{-4cm}Représentation de chaînes de caractères sous forme de nombres}
\author{Christophe Dronne}
\date{25 Octobre, 2023}

\begin{document}
\maketitle

\section*{Préliminaires}
On définit une base quelconque, notée $b$, utilisant comme symboles les lettres de notre alphabet en plus d'autres symboles utilisées comme caractères typographiques. Cette base servira à représenter un nombre sous forme de texte, de la même manière que le nombre $(13303790)_{10}$ peut être représenté sous le nombre hexadécimal $($caffee$)_{16}$ qui pourrait être interprété comme le mot "caffée".

Une implémentation possible et simple de cette base pourrait utiliser les lettres "a", "b", "c" ... "z" pour représenter les nombres $0$, $1$, $2$, ... $25$, avec $b = 26$. Cette base va être utilisée pour illustrer des exemples, mais de manière générale tout formule et propriété va être applicable à une toute base supérieure ou égale à 2.

Le but de cet exercice est de trouver une implémentation optimale des fonctions communes utilisées pour les chaînes de caractères en informatique avec ces nombres en base $b$. \\

\noindent Lors de cet exercice, on va souvent faire utilisation de la fonction de la partie entière, notée $E(x)$, et de la fonction de la partie fractionnaire, notée
$frac(x)$, tel: \\

$\forall x \in \mathbb{R}$, on a:

\begin{equation}
E \left( x \right) \in \mathbb{N}
\end{equation}

\begin{equation}
frac \left( x \right) = | x | - E \left( | x | \right)
\end{equation} \\

Or, on prendra seulement en compte des valeurs positives, donc:

\begin{equation}
frac \left( x \right) = x - E \left( x \right)
\end{equation}

\newpage

\noindent On a les propriétés suivantes:

\begin{align*}
0 \leqslant frac \left( x \right) < 1

\Rightarrow E \left( frac \left( x \right) \right) = 0
\end{align*} \\ \\

\begin{align*}
0 \leqslant x < 1

\Rightarrow frac \left( x \right) = x

\Rightarrow E \left( x \right) = E \left( frac \left( x \right) \right) = 0
\end{align*} \\ \\

\begin{align*}
$frac$ \left( frac \left( x \right) \right) = frac \left( x \right) -
E \left( frac \left( x \right) \right)

\Rightarrow frac \left( frac \left( x \right) \right) =
frac \left( x \right) - 0

\Rightarrow frac \left( frac \left( x \right) \right) = frac \left( x \right)
\end{align*} \\ \\

$\forall \left( x_{1}; x_{2} \right) \in \mathbb{R}^{2}$, on a:

\begin{align*}
$E$ \left( x_{1} + x_{2} \right) = E \left( x_{1} \left) + E \left( x_{2} \right)

si, et seulement si:

0 \leqslant frac \left( x_{1} \right) + frac \left( x_{2} \right) < 1

sinon:

$E$ \left( x_{1} + x_{2} \right) = E \left( x_{1} \left) + E \left( x_{2} \right) + 1
\end{align*} \\ \\

Donc, $\forall x \in \mathbb{R}, \forall n \in \mathbb{N}$, on a:

\begin{align*}
$E$ \left( x + n \right) = E \left( x \right) + n

car:

$frac$ \left( x \right) + frac \left( n \right) =
frac \left( x \right) + 0 = frac \left( x \right)

et

0 \leqslant frac \left( x \right) < 1

Donc $E$ \left( x + n \right) = E \left( x \right) + E \left( n \right)

De plus, on a:
E \left( n \right) = n$, donc$

$E$ \left( x + n \right) $ est bien égal à $  E \left( x \right) + n


\end{align*}

\newpage

\section*{Longueur d'un nombre}
Les chaînes de caractères ont toujours une longueur associée. Pour donner un exemple, la chaîne "bonjour" a une longueur de $7$, car elle est constituée de $7$ caractères. Cependant, le nombre $($bonjour$)_{b}$ n'a pas de 'longueur', donc il va falloir trouver une définition mathématique de la longueur d'un nombre. \\

\noindent On pose:

\begin{align*}
\forall b \in \mathbb{N} - \{0; 1\} $, une base quelconque$

\forall n \in \mathbb{N} $, la quantité de chiffres dans $ k

\forall k \in \mathbb{R} $, tel que $ 1 \leqslant k < b
\end{align*} \\

\noindent Tout entier naturel $l$ (exclu de 0) de base $b$ peut s'écrire tel:

$l = k \times b^{n - 1}$ \\

\noindent \textbf{Exemple}:

$534 = 5.34 \times 10^{3 - 1}$, avec $b = 10$, $n = 3$, $l = 534$, $k = 5.34$,

suivant les conditions posées.

\noindent On pose la formule de la 'longueur' (c-à-d quantité de chiffres) d'un nombre:

\begin{equation}
n = E \left( log_{b} \left( l \right) \right) + 1
\end{equation}

\noindent Par convénience, on va noter la fonction qui attribue à un entier naturel non nul sa quantité de chiffres $len(l)$, tel que:

\begin{equation}
len(l) = E \left( log_{b} \left( l \right) \right) + 1
\end{equation}

\noindent \textbf{Démonstration}:

\noindent On pose:

\begin{align*}
\forall b \in \mathbb{N} - \{0; 1\} $, une base quelconque$

\forall n \in \mathbb{N} $, la quantité de chiffres dans $ k

\forall k \in \mathbb{R} $, tel que $1 \leqslant k < b
\end{align*} \\

\noindent On a:

\begin{align*}
1 \leqslant k < b

$k$ et $b$ sont positifs, donc:

\Rightarrow log_{b} \left( 1 \right) \leqslant log_{b} \left( k \right) < log_{b} \left( b \right)

\Rightarrow 0 \leqslant log_{b} \left( k \right) < 1

\Rightarrow E \left( \log_{b} \left( k \right) \right) = 0
\end{align*}

\newpage

\noindent On reprend la formule de $l$:

\begin{align*}
$l = k \times b^{n - 1}$

\Rightarrow log_{b} \left( l \right) = log_{b} \left( k \times b^{n - 1} \right)

\Rightarrow E \left( log_{b} \left( l \right) \right) =
E \left( log_{b} \left( k \times b^{n - 1} \right) \right)

\Rightarrow E \left( log_{b} \left( l \right) \right) =
E \left( log_{b} \left( k \right) + log_{b} \left( b^{n-1} \right) \right)

\Rightarrow  E \left( log_{b} \left( l \right) \right) =
E \left( log_{b} \left( k \right) + n - 1 \right)
\end{align*} \\

$(n - 1) \in \mathbb{N}$, et $\forall (n_{1}; n_{2}) \in \mathbb{N}^{2}$,
$\left( n_{1} + n_{2} \right) \in \mathbb{N}$,

donc $\forall x \in \mathbb{R}$, $E \left( x + n \right) = E \left( x \right) + n$,
donc: \\

\begin{align*}
\Rightarrow  E \left( log_{b} \left( l \right) \right) =
E \left( log_{b} \left( k \right) \right) + n - 1

\Rightarrow  E \left( log_{b} \left( l \right) \right) + 1 =
E \left( log_{b} \left( k \right) \right) + n

or, on sait que $E \left( log_{b} \left( k \right) \right) = 0$, donc:

\Rightarrow  E \left( log_{b} \left( l \right) \right) + 1 = 0 + n
\end{align*} \\

\noindent On retrouve bien:

$n = E \left( log_{b} \left( l \right) \right) + 1$ \\

\noindent On a donc vérifié l'équation et justifié la validité de la fonction $len(l)$. \\

\noindent \textbf{Exemple}:

\noindent On reprend la base $b = 26$ définie dans les préliminaires:

\begin{align*}
$l = ($bonjour$)_{26} = (481363913)_{10}$

$len$ \left( l \right) = E \left( log_{26} \left( l \right) \right) + 1

$log$_{26} \left( l \right) \approx 6.1361

E \left( log_{26} \left( l \right) \right) = 6

$len$ \left( l \right) = E \left( log_{26} \left( l \right) \right) + 1 = 6 + 1

$len$ \left( l \right) = 7
\end{align*} \\

\noindent En comptant la quantité de chiffres dans $($bonjour$)_{26}$, on retrouve bien $7$.

\newpage

\section*{Indice d'un nombre}
On s'intéresse à trouver le chiffre d'indice $n \in \mathbb{N}$ d'un entier naturel quelconque $l$ représenté en base quelconque $b$. On note la fonction qui permet de trouver ce chiffre $ind(l; n)$. \\

\noindent On pose:

\begin{equation}
ind(l;n) = E \left( \frac{l}{b^{n}} \right) -
E \left( \frac{l}{b^{n + 1}} \right) \times b
\end{equation}

On note que il s'agit de l'indice \textbf{partant de la droite} à $n = 0$. Pour avoir le chiffre d'indice $n$ \textbf{partant de la gauche}, il suffit de prendre $len(l) - n$ comme indice, soit $ind \left( l; len(l) - n \right)$, ce qui permet de suivre les conventions informatiques. \\

\noindent \textbf{Démonstration}:

\noindent On pose:

\begin{align*}
\forall x \in \mathbb{N}

\forall n \in \mathbb{N} $, l'indice du nombre recherché, tel que $ 0 \leqslant n < len(x) \end{align*} \\

\noindent On peut écrire $x$ tel:

\begin{equation}
x = \varepsilon_{1} \times b^{n + 1} + c \times b^{n} + \varepsilon_{2}
\end{equation}

\noindent avec:

\begin{align*}
0 \leqslant c < b $; $ c \in \mathbb{N}

0 \leqslant \varepsilon_{1} $; $ \varepsilon_{1} \in \mathbb{N}

0 \leqslant \varepsilon_{2} < b^{n} $; $ \varepsilon_{2} \in \mathbb{N}
\end{align*} \\

\noindent On cherche à trouver une formule de $c$ en fonction de $x$, $n$, et $b$. \\

\begin{align*}
\[ c = \frac{x - \varepsilon_{1} \times b^{n+1} - \varepsilon_{2}}{b^{n}} \]

\Rightarrow c = \frac{x - \varepsilon_{2}}{b^{n}} - \varepsilon_{1} \times b

\Rightarrow c + \varepsilon_{1} \times b = \frac{x - \varepsilon_{2}}{b^{n}}

\Rightarrow c + \varepsilon_{1} \times b = \frac{x}{b^{n}} - \frac{\varepsilon_{2}}{b^{n}}
\end{align*} \\

\newpage

\begin{align*}
\left( c + \varepsilon_{1} \times b^{n+1} \right) \in \mathbb{N} $, donc $
c + \varepsilon_{1} \times b^{n+1} = E \left( c + \varepsilon_{1} \times b^{n+1} \right)

\Rightarrow c + \varepsilon_{1} \times b^{n+1} =
E \left( \frac{x}{b^{n}} - \frac{\varepsilon_{2}}{b^{n}} \right)

\Rightarrow frac \left( \frac{x}{b^{n}} \right) =
frac \left( \frac{\varepsilon_{1} \times b^{n + 1} + c \times b^{n} + \varepsilon_{2}}{b^{n}} \right)

\Rightarrow frac \left( \frac{x}{b^{n}} \right) =
frac \left( \varepsilon_{1} \times b + c + \frac{\varepsilon_{2}}{b^{n}} \right)
\end{align*} \\

$(\varepsilon_{1} \times b + c) \in \mathbb{N}$, donc:

\begin{align*}
$frac$ \left( \varepsilon_{1} \times b + c \right) = 0

\Rightarrow frac \left( \varepsilon_{1} \times b + c + \frac{\varepsilon_{2}}{b^{n}} \right) =
frac \left( \frac{\varepsilon_{2}}{b^{n}} \right)

\Rightarrow frac \left( \frac{x}{b^{n}} \right) =
frac \left( \frac{\varepsilon_{2}}{b^{n}} \right)

\Rightarrow frac \left( \frac{x}{b^{n}} \right) -
frac \left( \frac{\varepsilon_{2}}{b^{n}} \right) < 1
\end{align*} \\

On reprend la formule de $\left( c + \varepsilon_{1} \times b^{n+1} \right)$:

\begin{align*}
c + \varepsilon_{1} \times b^{n+1} =
E \left( \frac{x}{b^{n}} - \frac{\varepsilon_{2}}{b^{n}} \right)

\Rightarrow c + \varepsilon_{1} \times b^{n+1} =
E \left( \frac{x}{b^{n}} \right) - E \left( \frac{\varepsilon_{2}}{b^{n}} \right)
\end{align*} \\

Or, on a:

\begin{align*}
0 \leqslant \varepsilon_{2} < b^{n}

\Rightarrow 0 \leqslant \frac{\varepsilon_{2}}{b^{n}} < \frac{b^{n}}{b^{n}}

\Rightarrow 0 \leqslant \frac{\varepsilon_{2}}{b^{n}} < 1

\Rightarrow E \left( \frac{\varepsilon_{2}}{b^{n}} \right) = 0
\end{align*} \\

\begin{align*}
c + \varepsilon_{1} \times b^{n+1} =
E \left( \frac{x}{b^{n}} \right) - E \left( \frac{\varepsilon_{2}}{b^{n}} \right)

\Rightarrow c + \varepsilon_{1} \times b^{n+1} =
E \left( \frac{x}{b^{n}} \right) - 0

\Rightarrow c + \varepsilon_{1} \times b^{n+1} = E \left( \frac{x}{b^{n}} \right)
\end{align*} \\

\[
c + \varepsilon_{1} \times b^{n+1} = E \left( \frac{x}{b^{n}} \right)
\]

Il suffit plus qu'à exprimer $\varepsilon_{1} \times b^{n+1}$ en fonction de $x$, $n$ et $b$.

\[
\varepsilon_{1} = \frac{x - c \times b^{n} - \varepsilon_{2}}{b^{n+1}}
\]

$\Rightarrow \varepsilon_{1} =
\frac{x}{b^{n+1}} - \frac{c}{b} - \frac{\varepsilon_{2}}{b^{n+1}}$ \\

$\varepsilon_{1} \in \mathbb{N}$, donc $\varepsilon_{1} = E \left( \varepsilon_{1} \right)$

$\Rightarrow \varepsilon_{1} =
E \left( \frac{x}{b^{n+1}} - \frac{c}{b} - \frac{\varepsilon_{2}}{b^{n+1}} \right)$ \\

\newpage

Or, on a:

$frac \left( \frac{x}{b^{n+1}} \right) =
frac \left( \frac{\varepsilon_{1} \times b^{n + 1} + c \times b^{n} + \varepsilon_{2}}{b^{n+1}} \right)$

$\Rightarrow frac \left( \frac{x}{b^{n+1}} \right) =
frac \left( \varepsilon_{1} + \frac{c}{b} + \frac{\varepsilon_{2}}{b^{n+1}} \right)$

$\Rightarrow frac \left( \frac{x}{b^{n+1}} \right) =
frac \left( \frac{c}{b} + \frac{\varepsilon_{2}}{b^{n+1}} \right)$, car
$\varepsilon_{1} \in \mathbb{N}$ \\

Donc:

$frac \left( \frac{x}{b^{n+1}} \right) -
frac \left( \frac{c}{b} \right) - 
frac \left( \frac{\varepsilon_{2}}{b^{n+1}} \right) = 0$ \\

Donc, on peut décomposer la formule:

$\varepsilon_{1} =
E \left( \frac{x}{b^{n+1}} - \frac{c}{b} - \frac{\varepsilon_{2}}{b^{n+1}} \right)$

$\Rightarrow \varepsilon_{1} =
E \left( \frac{x}{b^{n+1}} \right) -
E \left( \frac{c}{b} \right) - 
E \left( \frac{\varepsilon_{2}}{b^{n+1}} \right)$ \\

De plus, on a:

$b^{n+1} > c \times b^{n} > \varepsilon_{2} \geqslant 0$ (d'après leur définition)

$\Rightarrow 1 > \frac{c}{b} > \frac{\varepsilon_{2}}{b^{n+1}} \geqslant 0$

$\Rightarrow E \left( \frac{c}{b} \right) = 0$

$\Rightarrow E \left( \frac{\varepsilon_{2}}{b^{n+1}} \right) = 0$ \\

Donc:

$\varepsilon_{1} =
E \left( \frac{x}{b^{n+1}} \right) -
E \left( \frac{c}{b} \right) - 
E \left( \frac{\varepsilon_{2}}{b^{n+1}} \right)$

$\Rightarrow \varepsilon_{1} = E \left( \frac{x}{b^{n+1}} \right) - 0 - 0$

\[
\varepsilon_{1} = E \left( \frac{x}{b^{n+1}} \right)
\]

\noindent On a trouvé une formule pour $\varepsilon_{1}$ en fonction de $x$, $n$ et $b$. On note:

\[
c + \varepsilon_{1} \times b^{n+1} = E \left( \frac{x}{b^{n}} \right)
\]

\[
\Rightarrow c = E \left( \frac{x}{b^{n}} \right) - E \left( \frac{x}{b^{n+1}} \right) \times b^{n+1}
\]

\end{document}
