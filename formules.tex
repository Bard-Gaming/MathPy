\documentclass[a4paper, 12pt]{article}
\usepackage{amsfonts}
\usepackage{amssymb}

\title{\vspace{-4cm}Représentation de chaînes de caractères sous forme de nombres}
\author{Christophe Dronne}
\date{25 Octobre, 2023}

\begin{document}
\maketitle

\section*{Préliminaires}
On définit une base quelconque, notée $b$, utilisant comme symboles les lettres de notre alphabet en plus d'autres symboles utilisées comme caractères typographiques. Cette base servira à représenter un nombre sous forme de texte, de la même manière que le nombre $(13303790)_{10}$ peut être représenté sous le nombre hexadécimal $($caffee$)_{16}$ qui pourrait être interprété comme le mot "caffée".

Une implémentation possible et simple de cette base pourrait utiliser les lettres "a", "b", "c" ... "z" pour représenter les nombres $0$, $1$, $2$, ... $25$, avec $b = 26$. Cette base va être utilisée pour illustrer des exemples, mais de manière générale tout formule et propriété va être applicable à une toute base supérieure ou égale à 2.

Le but de cet exercice est de trouver une implémentation optimale des fonctions communes utilisées pour les chaînes de caractères en informatique avec ces nombres en base $b$.

\section*{Longueur d'un nombre}
Les chaînes de caractères ont toujours une longueur associée. Pour donner un exemple, la chaîne "bonjour" a une longueur de $7$, car elle est constituée de $7$ caractères. Cependant, le nombre $($bonjour$)_{b}$ n'a pas de 'longueur', donc il va falloir trouver une définition mathématique de la longueur d'un nombre. \\

\noindent On pose:

\begin{align*}
\forall b \in \mathbb{N} - \{0; 1\} $, une base quelconque$

\forall n \in \mathbb{N} $, la quantité de chiffres dans $ k

\forall k \in \mathbb{R} $, tel que $ 1 \leqslant k < b
\end{align*} \\

\noindent Tout entier naturel $l$ (exclu de 0) de base $b$ peut s'écrire tel:

$l = k \times b^{n - 1}$ \\

\noindent Exemple:

$534 = 5.34 \times 10^{3 - 1}$, avec $b = 10$, $n = 3$, $l = 534$, $k = 5.34$,

suivant les conditions posées.

\newpage

\noindent On pose la formule de la 'longueur' (c-à-d quantité de chiffres) d'un nombre:

\begin{equation}
n = E \left( log_{b} \left( l \right) \right) + 1
\end{equation}

\noindent avec $E(x)$ la fonction de la partie entière. \\

\noindent Par convénience, on va noter la fonction qui attribue à un entier naturel non nul sa quantité de chiffres $len(l)$, tel que:

\begin{equation}
len(l) = E \left( log_{b} \left( l \right) \right) + 1
\end{equation}

\noindent Démonstration:

On pose:

\begin{align*}
\forall b \in \mathbb{N} - \{0; 1\} $, une base quelconque$

\forall n \in \mathbb{N} $, la quantité de chiffres dans $ k

\forall k \in \mathbb{R} $, tel que $1 \leqslant k < b
\end{align*} \\

On a:

\begin{align*}
1 \leqslant k < b

$k$ et $b$ sont positifs, donc:

\Rightarrow log_{b} \left( 1 \right) \leqslant log_{b} \left( k \right) < log_{b} \left( b \right)

\Rightarrow 0 \leqslant log_{b} \left( k \right) < 1

\Rightarrow E \left( \log_{b} \left( k \right) \right) = 0
\end{align*} \\

On reprend la formule de $l$:

\begin{align*}
$l = k \times b^{n - 1}$

\Rightarrow log_{b} \left( l \right) = log_{b} \left( k \times b^{n - 1} \right)

\Rightarrow E \left( log_{b} \left( l \right) \right) =
E \left( log_{b} \left( k \times b^{n - 1} \right) \right)

\Rightarrow E \left( log_{b} \left( l \right) \right) =
E \left( log_{b} \left( k \right) + log_{b} \left( b^{n-1} \right) \right)

\Rightarrow  E \left( log_{b} \left( l \right) \right) =
E \left( log_{b} \left( k \right) + n - 1 \right)
\end{align*} \\

$(n - 1) \in \mathbb{N}$, et $\forall (n_{1}; n_{2}) \in \mathbb{N}^{2}$,
$\left( n_{1} + n_{2} \right) \in \mathbb{N}$,

donc $\forall x \in \mathbb{R}$, $E \left( x + n \right) = E \left( x \right) + n$,
donc: \\

\begin{align*}
\Rightarrow  E \left( log_{b} \left( l \right) \right) =
E \left( log_{b} \left( k \right) \right) + n - 1

\Rightarrow  E \left( log_{b} \left( l \right) \right) + 1 =
E \left( log_{b} \left( k \right) \right) + n

or, on sait que $E \left( log_{b} \left( k \right) \right) = 0$, donc:

\Rightarrow  E \left( log_{b} \left( l \right) \right) + 1 = 0 + n
\end{align*} \\

On retrouve bien:

$n = E \left( log_{b} \left( l \right) \right) + 1$

\newpage

\noindent Exemple:

\begin{align*}
$l = ($bonjour$)_{25} = (481363913)_{10}$

$len$ \left( l \right) = E \left( log_{25} \left( l \right) \right) + 1

$log$_{25} \left( l \right) \approx 6.2109056

E \left( log_{25} \left( l \right) \right) = 6

$len$ \left( l \right) = E \left( log_{25} \left( l \right) \right) + 1 = 7

$len$ \left( l \right) = 7
\end{align*} \\

En comptant la quantité de chiffres dans $($bonjour$)_{25}$, on retrouve bien $7$.

\section*{Indice d'un nombre}

\end{document}
